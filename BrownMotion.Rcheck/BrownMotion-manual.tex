\nonstopmode{}
\documentclass[letterpaper]{book}
\usepackage[times,inconsolata,hyper]{Rd}
\usepackage{makeidx}
\usepackage[utf8,latin1]{inputenc}
% \usepackage{graphicx} % @USE GRAPHICX@
\makeindex{}
\begin{document}
\chapter*{}
\begin{center}
{\textbf{\huge Package `BrownMotion'}}
\par\bigskip{\large \today}
\end{center}
\begin{description}
\raggedright{}
\item[Version]\AsIs{1.1}
\item[Author]\AsIs{Peigen Zhou}
\item[Maintainer]\AsIs{Peigen Zhou	}\email{peigen@stat.wisc.edu}\AsIs{}
\item[License]\AsIs{MIT}
\item[Title]\AsIs{BrownMotion for homework}
\item[Description]\AsIs{....as you see}
\item[URL]\AsIs{}\url{https://github.com/BaconZhou/BrownMotion}\AsIs{}
\item[Suggests]\AsIs{knitr, rmarkdown}
\item[VignetteBuilder]\AsIs{knitr}
\item[NeedsCompilation]\AsIs{no}
\end{description}
\Rdcontents{\R{} topics documented:}
\inputencoding{utf8}
\HeaderA{BrownMotion}{BrownMotion test for R package building process}{BrownMotion}
%
\begin{Description}\relax
BrownMotion test for R package building process
\end{Description}
%
\begin{Usage}
\begin{verbatim}
BrownMotion(n = 100, distribution = "normal", sigma = 1, dim = 2)
\end{verbatim}
\end{Usage}
%
\begin{Arguments}
\begin{ldescription}
\item[\code{n}] Number of data.

\item[\code{distribution}] "normal" or "T" represent normal distribution or T distribution.

\item[\code{sigma}] standard deviation for normal distribution and t distribution.

\item[\code{dim}] dimension for dataset.
\end{ldescription}
\end{Arguments}
%
\begin{Value}
return the dataset you want
\end{Value}
%
\begin{Examples}
\begin{ExampleCode}
y <- BrownMotion(10000)
plot(y,type='l')
\end{ExampleCode}
\end{Examples}
\printindex{}
\end{document}
